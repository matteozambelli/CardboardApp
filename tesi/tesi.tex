%----------------------------------------------------------------------------------------
%	PACKAGES AND OTHER DOCUMENT CONFIGURATIONS
%----------------------------------------------------------------------------------------

\documentclass[12pt,a4paper,twoside]{book}
\usepackage{times}
\usepackage[latin1]{inputenc}
\usepackage[italian]{babel}
\usepackage{amsmath}
\usepackage{amsfonts}
\usepackage{amssymb}
\usepackage{graphicx}
\usepackage{setspace}
\usepackage{fancyhdr}
\usepackage{emptypage}
\renewcommand{\baselinestretch}{1.5} 
%sommario
%profondit� 1 = sezioni (2 = sottosezioni ecc..)
\setcounter{tocdepth}{1}
\onehalfspacing
%numeri di pagina dispari a dx e pari a sx nel pi� di pagina
\pagestyle{fancy}
\fancyhf{}
\fancyfoot[LE,RO]{\thepage}
\fancypagestyle{plain}{\fancyfoot[RO,LE]{\thepage}}


%----------------------------------------------------------------------------------------
%	TITLE SECTION
%----------------------------------------------------------------------------------------

\author{Fabio Terzi, Matteo Zambelli} % Your name

\begin{document}
	
\tableofcontents % l'indice


%----------------------------------------------------------------------------------------
%	introduzione
%----------------------------------------------------------------------------------------

\chapter{Introduzione}

Il progetto 3D4Amb mira a sviluppare un sistema basato sul 3D per la diagnosi e il trattamento dell?ambliopia nei bambini piccoli.\\
Sfrutta la tecnologia 3D active shutter per garantire una visione binoculare, cio� per mostrare immagini diverse all'occhio normale e all'occhio pigro. Essa dovrebbe consentire una facile diagnosi dell'ambliopia e il suo trattamento per mezzo di giochi interattivi e attivit� di intrattenimento. Non dovrebbe soffrire dei problemi del trattamento classico dell'occlusione, � adatto ad un uso domestico, e potrebbe, almeno in parte, sostituire l'occlusione dell'occhio normale.\\                           
L'obiettivo principale di questo progetto di ricerca, denominato 3D4Amb, � di sviluppare un sistema per la diagnosi e il trattamento di ambliopia, basata sulla visione binoculare in modo accessibile. Con il termine accessibile si intende: poco costoso, user friendly, adatto per uso domestico e facilmente estendibile.\\
Tutte le informazioni sul progetto sono reperibili sul sito: http://3d4amb.unibg.it/
Car Racing Cardboard � un'applicazine per la piattaforma Android, il suo scopo � curare una patologia come l?ambliopia attraverso un gioco, in modo tale da far divertire il paziente ed allo stesso tempo sottoporlo al trattamento per la cura della sua malattia.

%----------------------------------------------------------------------------------------
%	l'ambliopia
%----------------------------------------------------------------------------------------

\chapter{L'Ambliopia}
\section{Il disturbo}
L'ambliopia � una condizione di ridotta acuit� visiva mono o bilaterale e si manifesta indipendentemente da causa organica. � dovuta ad una inadeguata stimolazione visiva
durante il periodo plastico del sistema visivo, ossia il periodo che va dalla nascita fino ai sette anni.\\
Il soggetto in cui � presente l'ambliopia soffre di un alterazione della visione dello spazio: le immagini che provengono dagli occhi non vengono correttamente rielaborate all'interno del cervello. Questo causa una scorretta comprensione dello spazio che lo circonda e causa una percezione scorretta della profondit�, dei movimenti e dei contrasti.
� presente nel 2-4\% della popolazione, la sua incidenza � pi� elevata in associazione con alcune condizioni quali prematurit�, sindrome di Down, patologia neurologica e familiarit� per ambliopia o strabismo. Spesso le persone non si accorgono nemmeno di esserne affette fino ai 20-30 anni, per questo � fondamentale la diagnosi.\\
Pu� colpire i bambini dalla nascita fino ai 7 anni, et� in cui il sistema visivo raggiunge la sua maturit�. Durante questo periodo iniziale l?ambliopia pu� essere trattata e prevenuta, mentre superata questa fase l?istaurazione della malattia diventa impossibile, ma, nel caso fosse presente, essa risulta irreversibile. L?ambliopia funzionale deve essere distinta dall?ambliopia organica, la quale � un impoverimento della visione, causata da anomalie strutturali dell?occhio o del cervello, che sono indipendenti dagli input sensoriali.\\
L'ambliopia funzionale � reversibile se trattata con la stimolazione visiva adeguata, mentre quella organica non subisce alcun beneficio da una stimolazione visiva.
Il videogame di rebalance ha quindi effetto solo sull' ambliopia funzionale e non su quella organica.

%----------------------------------------------------------------------------------------
%	il prinicipio del trattamento
%----------------------------------------------------------------------------------------

\chapter{Il principio del trattamento tramite il gioco}

%----------------------------------------------------------------------------------------
%	l'applicazione
%----------------------------------------------------------------------------------------

\chapter{Car racing cardboard: l'applicazione}

%----------------------------------------------------------------------------------------
%	raccolta dati
%----------------------------------------------------------------------------------------

\chapter{??Raccolta dati??}

%----------------------------------------------------------------------------------------
%	social media
%----------------------------------------------------------------------------------------

\chapter{Social media}

%----------------------------------------------------------------------------------------
%	paper
%----------------------------------------------------------------------------------------

\chapter{Paper}

\end{document}